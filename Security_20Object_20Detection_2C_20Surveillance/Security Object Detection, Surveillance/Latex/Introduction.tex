\section{Introduction}

Security and safety have always been a great issue, making it a great benefit to implement security features in the place of comfort. Anyone would agree that such technologies that can solve security problems could be a great asset to many lives. Using computer vision and neural networks, we can introduce a convenient solution to the lack of safety of any individual’s home.

Computer vision and neural network technologies are based on image and video recognition using convoluted connected layers of a variety of dimensions \cite{Maladkar}. Using a coded algorithm, the data can be calculated to inform the user of the percentage of identification of the object in front of the camera recording. For example, if you put a coca-cola can in front of the camera it should be able to at least inform the user of it being 60 percent a can, etc.
Such designs can cause advantages in security because it allows the user to be warned of the object that may have disrupted the safety of their home. If an ordered item from Amazon Co. was considered “delivered” but it was not found in the descriptive delivery information, the program should be able to identify the situation of it being taken by an animal or a person. This is a great advantage to the user because if the theft was greatly hidden, any slight picture can be identified by the program to ensure the percentages of the object, making the program very beneficial.

Our objective is to develop a computer vision program, embedded in a raspberry pi to use it with a variety of applications. In our case, we will implement it for a security system for homes. The main objective is to layout an image recognition software to tackle this problem embedded in raspberry pi using Python, TensorFlow, Linux, and coco. Tensorflow is an open-sourced software developed by Google, that allows us to use COCO and is available for Linux, another open-source operating system. 
 