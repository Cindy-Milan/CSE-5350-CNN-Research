\section{Conclusion}
Computer vision and neural networks are mainly used as security surveillance for the safety of homes and companies. This goes from at-home safety to labor worker safety in construction sites. The image detector uses a variety of layers and boxes with different pixels, using linear algebra, the data is then calculated to identify the image with a percentage of certainty and the used frames per second based on the materials used.  There are many tutorials used to help create the surveillance device embedded in raspberry pi, using Linux, TensorFlow, and COCO. One uses MobileNetV3, while the other uses MobileNetV1, but due to staying under budget, the second tutorial was the best way to go since the first one required more materials. It uses a better camera and software system that can also be used for a more efficient algorithm. 

We also looked at different boards and found the most efficient way to go, that being a raspberry pi since it works with Linux, the operating system, and python, the programming language. Overall, we were successful in creating an algorithm that detects humans using existing COCO datasets and Tensorflow Lite to accomplish our goal. Furthermore, the algorithm can be modified and improved upon to increase its functionality and efficiency. the algorithm is able to detect humans with 70\% accuracy or better. Combined with different hardware, it has the potential of replacing different equipment in addition in aiding in home surveillance. We have also demonstrated how CNN can be used to detect other objects, not just humans.
